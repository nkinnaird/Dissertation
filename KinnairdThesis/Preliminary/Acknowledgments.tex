%!TEX root = ../thesis.tex


There are many people I would like to thank in my journey as a graduate student. First and foremost, my advisor Lee Roberts, for whom it has been an absolute pleasure and honor to work for. He gave me a special and unique opportunity to join the BU physics group and E989 collaboration, and I will always cherish it. To my second reader of this dissertation, Rob Carey, who has taught me many things along the way, including introducing me to the track fitting which was my first major project on the experiment. Thanks also to Jim Miller, who has always provided advice and suggestions regarding my work. Thanks to my non \gmtwo committee members Ken Lane and Shyam Erramilli, who taught me particle theory and electromagnetism respectively, without which I would not have understood the experiment I was on. Much thanks to Mirtha Cabello, the BU physics graduate student coordinator, who was always a joy to talk to and who was invaluable when it came to dealing with all of the administrative stuff. Thanks to my BU office buddy and friend, John Quirk, who was always a great source of humor and grounding when things got tough. I would also like to give special thanks to James Mott, who had endless patience and support for me as I grew, both as a graduate student and a person. He was a fantastic advisor to me as I lived in Fermilab these last few years, and being able to turn my chair around and ask for advice saved my graduate career on more than occasion.


I'd like to thank the entire E989 collaboration, without whom none of this work would have been possible. I'd first like to thank the Tracking team, of which I was a part for many years. Thanks to Brendan Casey, James Mott, Mark Lancaster, Will Turner, Saskia Charity, Joe Price, Barry King, Becky Chislett, Tabitha Halewood-Leagas, Talal Albahri, Tammy Walton, and Gleb Lukicov. I'm not sure there's a better group to have been a part of in all of particle physics to be honest. Special thanks again to Will and Gleb, who were my closest friends while at Fermilab, and were always a great source of support and camaraderie. Thanks also to Aaron Fienberg and David Sweigart, who were fantastic peers in the precession frequency analysis, and who I will always strive to emulate in the way I approach my work. Thanks also to Alex Keshavarzi, Sudeshna Ganguly, Anna Driutti, Leah Welty-Rieger, Midhat Farooq, Joe Grange, Jason Hempstead, Rachel Osofsky, James Stapleton, Nam Tran, Yaqian Wang, and many, many more who provided advice, help, and friendship along the way. 


I would also like to thank my undergraduate advisor Karol Lang and Marek Proga from the University of Texas at Austin, for whom I really got started in the field of particle physics many years ago.


Lastly, I'd like to thank my family. Thanks to my parents, John, Nett and Zindy, my sister Maris, and my brothers River and Aidan. I have received endless love and support from them over the years, and I am an extraordinarily lucky person to have the family that I do. Finally, to my love, Jenny Mahon, who has made the last year of my graduate school career the best one yet, and with whom I'm excited to share many future adventures with.


My participation in this work was supported in part by the U.S. Department of Science Office of High Energy Physics Award DE-SC0013895.
