%!TEX root = ../thesis.tex

\thispagestyle{myheadings}

\graphicspath{{Body/Figures/Wa/Datasets/Endgame/LostMuonFiles/MainCuts/}{Body/Figures/Wa/Datasets/ComparisonPlots/LostMuons/}{Body/Figures/Wa/Datasets/9d/SingleIteration/LostMuonFits/}{Body/Figures/Wa/Datasets/9d/PileupJobs/PileupGapTime/}{Body/Figures/Wa/Datasets/9d/PileupJobs/PileupDeadTime/auto-scaling/}{Body/Figures/Wa/Datasets/9d/PileupJobs/PileupDeadTime/fixed-scaling/}{Body/Figures/Wa/Datasets/9d/PileupJobs/PileupEnergyScale/}{Body/Figures/Wa/Datasets/9d/PileupJobs/PileupTimeShift/}{Body/Figures/Wa/Datasets/9d/SingleIteration/PileupMultiplierScan/}{Body/Figures/Wa/Datasets/60h/RatioConstruction/Ta/}{Body/Figures/Wa/Datasets/60h/RatioConstruction/TauMu/}{Body/Figures/Wa/Datasets/9d/Binning/BinEdge/}{Body/Figures/Wa/Datasets/9d/Binning/BinWidth/}}




\section{Systematic errors}
\label{sec:Systematic Errors}



\begin{table}[]
\centering
\setlength\tabcolsep{10pt}
\renewcommand{\arraystretch}{1.2}
\begin{tabular*}{.8\linewidth}{@{\extracolsep{\fill}}lc}
  \hline
    \multicolumn{2}{c}{\textbf{\wa Measurement Uncertainties}} \\
  \hline\hline
    Source of uncertainty & E989 Goal (ppb) \\
  \hline
    Gain changes & 20 \\
    Pileup & 40 \\
    Lost muons & 20 \\
    CBO & 30 \\
    E field and pitch corrections & 30 \\
  \hline
    Quadrature sum & 70 \\
  \hline 
\end{tabular*}
\caption[Uncertainties in the precession frequency measurement]{Systematic errors in the precession frequency measurement. \textbf{fill this table out more once I've gone through the various parts - mention that this is the final expected table}}
\label{tab:wauncertainties}
\end{table}





\subsection{Pileup systematic errors}
\label{sub:pileuperror}

As described in \secref{sub:pileupsubtraction}, the pileup background oscillates at \wa which by extension means a strong effect on the final fitted $R$ value. If the subtracted pileup spectrum is misconstructed in any way, there will be a systematic error on $R$. In general the pileup systematic error can be separated into two parts, the error on the amplitude and the error on the phase. In order to estimate the two parts, the uncertainties on the pileup amplitude and phase need to be estimated along with the sensitivities of $R$ to them. \tabref{tab:histogramparameters} gives the default values used for the pileup construction parameters \{ADT, SDT, SGT, C\}. How these parameters feed into the amplitude and phase systematic errors will be discussed in turn, and the overall errors calculated for the different Run~1 datasets.


As a reminder the default values used for the ADT and SDT were \ns{5} each, and a default automatic pileup amplitude multiplier of $\sim1.03$ was applied to the pileup spectra. In order to calculate the systematic dependence on the choice of ADT or SDT, the SDT parameter was scanned over from \ns{5} to \ns{10} in steps of \ns{1}. This was done with and without the same automatic pileup amplitude scaling procedure as described in \secref{sub:pileupsubtraction}. The results of the study for the 9d dataset are shown in Figures~\ref{fig:SDTscan_noScaling} and \ref{fig:SDTscan_autoScaling}. In the case where there was no automatic scaling applied there is a clear minimum in the \chisq results and a steep slope in $R$ corresponding to a large sensitivity of $R$ to the choice of SDT. In the case where the automatic scaling was applied however, the minimum in the \chisq results has disappeared, while the sensitivity of $R$ has become much reduced to the point of no longer being a clear trend\footnote{This slope in $R$ varies between positive and negative values based on dataset, so there is no real clear trend in $R$.}. The fact that applying the automatic pileup amplitude scaling produces nearly identical pileup spectra with no clear trend in $R$ regardless of the choice of SDT (and by extension ADT), any systematic error due to the choice of these two parameters can be subsumed into the direct pileup amplitude error itself, discussed down below. It should be noted in fact that the choices of ADT and SDT are largely irrelevant barring statistics, as the automatic amplitude scaling procedure can always account for any differences between the two. 


\begin{figure}[]
\centering
    \begin{subfigure}[t]{0.45\textwidth}
        \centering
        \includegraphics[width=\textwidth]{FullRatio_Chi2_Vs_ShadowDeadTime_Canv_9d_fixed}
        \caption{\chisq versus SDT. The parabolic fit equation used was $y = p_{2}(x - p_{1})^{2} + p_{0}.$}
    \end{subfigure}% %you need this % here to add spacing between subfigures
    \hspace{1cm}
    \begin{subfigure}[t]{0.45\textwidth}
        \centering
        \includegraphics[width=\textwidth]{FullRatio_R_Vs_ShadowDeadTime_Canv_9d_fixed}
        \caption{$R$ versus SDT. The parameter $p_{1}$ gives the sensitivity of $R$ to the value of SDT, with units in ppm/ns.}
    \end{subfigure}

    \begin{subfigure}[t]{0.45\textwidth}
        \centering
        \includegraphics[width=\textwidth]{SDT_PileupTimeComparison_9d_fixed}
        \caption{The pileup time spectrum for different choices of SDT.}
    \end{subfigure}% %you need this % here to add spacing between subfigures
    \hspace{1cm}
    \begin{subfigure}[t]{0.45\textwidth}
        \centering
        \includegraphics[width=\textwidth]{SDT_CorrEnergyComparison_9d_fixed}
        \caption{The corrected energy spectrum for different choices of SDT.}
    \end{subfigure}
\caption[Pileup shadow dead time scan without automatic pileup amplitude scaling]{Shadow dead time scan results without automatic pileup amplitude scaling. A clear minimum in the \chisq plot is seen near \ns{5} corresponding to the choice of ADT, and a large sensitivity for $R$ is observed. In the bottom two spectra plots the magenta curve corresponds to a choice SDT = \ns{5} while the black curve corresponds to SDT = \ns{10}. The larger choice of SDT leads to a greater estimation of the pileup, which as shown in the energy spectra plot leads to a corresponding over-subtraction at energies where hits consist mostly or purely of pileup pulses. Data from 9d dataset.}
\label{fig:SDTscan_noScaling}
\end{figure}


\begin{figure}[]
\centering
    \begin{subfigure}[t]{0.45\textwidth}
        \centering
        \includegraphics[width=\textwidth]{FullRatio_Chi2_Vs_ShadowDeadTime_Canv_9d_auto}
        \caption{\chisq versus SDT. The parabolic fit equation used was $y = p_{2}(x - p_{1})^{2} + p_{0}.$}
    \end{subfigure}% %you need this % here to add spacing between subfigures
    \hspace{1cm}
    \begin{subfigure}[t]{0.45\textwidth}
        \centering
        \includegraphics[width=\textwidth]{FullRatio_R_Vs_ShadowDeadTime_Canv_9d_auto}
        \caption{$R$ versus SDT. The parameter $p_{1}$ gives the sensitivity of $R$ to the value of SDT, with units in ppm/ns.}
    \end{subfigure}

    \begin{subfigure}[t]{0.45\textwidth}
        \centering
        \includegraphics[width=\textwidth]{SDT_PileupTimeComparison_9d_auto}
        \caption{The pileup time spectrum for different choices of SDT.}
    \end{subfigure}% %you need this % here to add spacing between subfigures
    \hspace{1cm}
    \begin{subfigure}[t]{0.45\textwidth}
        \centering
        \includegraphics[width=\textwidth]{SDT_CorrEnergyComparison_9d_auto}
        \caption{The corrected energy spectrum for different choices of SDT.}
    \end{subfigure}
\caption[Pileup shadow dead time scan with automatic pileup amplitude scaling]{Shadow dead time scan results with automatic pileup amplitude scaling. No clear minimum is observed in the \chisq plot, and the sensitivity for $R$ is small. In the bottom two spectra plots the magenta curve corresponds to a choice SDT = \ns{5} while the black curve corresponds to SDT = \ns{10}. With the automatic amplitude scaling applied, the time and energy spectra are nearly identical and lie on top of each other. That combined with the lack of clear minimum in the \chisq plot and no clear sensitivity in $R$ indicate that there is no real systematic error due to the choice of SDT. Data from 9d dataset.}
\label{fig:SDTscan_autoScaling}
\end{figure}


In order to calculate the systematic dependence on the choice of SGT (default value of \ns{10}), the SGT parameter was scanned over from \ns{10} to \ns{20} in steps of \ns{1}. The results of the study with the automatic pileup amplitude scaling applied is shown in \figref{fig:SGTscan}. Just as in the SDT scan with the automatic pileup amplitude scaling, there is no minimum in the \chisq results, the sensitivity of $R$ to the value of SGT not so clear, and the pileup spectra for the various choices of SGT are nearly identical. Therefore again any systematic error due to the choice of SGT is subsumbed into the pileup amplitude error.


\begin{figure}[]
\centering
    \begin{subfigure}[t]{0.45\textwidth}
        \centering
        \includegraphics[width=\textwidth]{FullRatio_Chi2_Vs_ShadowGapTime_Canv_9d}
        \caption{\chisq versus SGT. The parabolic fit equation used was $y = p_{2}(x - p_{1})^{2} + p_{0}.$}
    \end{subfigure}% %you need this % here to add spacing between subfigures
    \hspace{1cm}
    \begin{subfigure}[t]{0.45\textwidth}
        \centering
        \includegraphics[width=\textwidth]{FullRatio_R_Vs_ShadowGapTime_Canv_9d}
        \caption{$R$ versus SGT. The parameter $p_{1}$ gives the sensitivity of $R$ to the value of SGT, with units in ppm/ns.}
    \end{subfigure}

    \begin{subfigure}[t]{0.45\textwidth}
        \centering
        \includegraphics[width=\textwidth]{SGT_PileupTimeComparison_9d}
        \caption{The pileup time spectrum for different choices of SGT.}
    \end{subfigure}% %you need this % here to add spacing between subfigures
    \hspace{1cm}
    \begin{subfigure}[t]{0.45\textwidth}
        \centering
        \includegraphics[width=\textwidth]{SGT_CorrEnergyComparison_9d}
        \caption{The corrected energy spectrum for different choices of SGT.}
    \end{subfigure}
\caption[Pileup shadow gap time scan with automatic pileup amplitude scaling]{Shadow gap time scan results with automatic pileup amplitude scaling. No clear minimum is observed in the \chisq plot, and the trend for $R$ isn't clear, with points fluctuating above and below the fit curve. In the bottom two spectra plots one of the grey curves (hidden) corresponds to a choice SGT = \ns{10} while the black curve corresponds to SGT = \ns{20}. With the automatic amplitude scaling applied, the time and energy spectra lie on top of each other. That combined with the lack of clear minimum in the \chisq plot and small sensitivity in $R$ indicate that there is no real systematic error due to the choice of SGT. Data from 9d dataset.}
\label{fig:SGTscan}
\end{figure}


The pileup amplitude systematic error is the error on $R$ assuming the scale of the pileup was incorrectly constructed. In order to evaluate this error, multipliers were applied to the pileup spectra from 0.9 to 1.1 in steps of 0.01 (dropping the default automatic pileup scaling of $\sim1.03$ mentioned before). The data was then re-fit to find the change in $R$. The results of the study for the 9d dataset are shown in \figref{fig:PMscan}. As shown there is a clear minimum near 1 in the \chisq results and a large sensitivity of $R$ to the multiplier. The systematic error on $R$ is calculated as 
    \begin{align}
        \delta R = \sigma_{P_{m}} \times \frac{dR}{dP_{m}},
    \end{align}
where $P_{m}$ is the value of the pileup multiplier. The error $\sigma_{P_{m}}$ is calculated as the width of the fitted parabola in the \chisq plot, defined as the change in $P_{m}$ from the minimum for the \chisq to increase by 1. This is calculated as 
    \begin{align}
        \sigma_{P_{m}} = \sqrt{\frac{2}{f''(\chi^{2})}} = \frac{1}{\sqrt{p_{2}}},
    \end{align}
where $p_{2}$ is the fit parameter as given in the top right of the \chisq plot. The sensitivities of $R$ to the pileup multiplier, uncertainties in the pileup amplitude, and final corresponding systematic errors for the Run~1 precession frequency analysis datasets are given in \tabref{tab:systematicError_pileupMultplier}. As shown in the table, the uncertainties on the pileup amplitude are of order 2 to 5\%, while the systematic errors on $R$ are on the order of \SI{10}{} to \SI{20}{ppb} depending on dataset.  It should be noted that the default automatic pileup multiplier of $\sim1.03$ does not necessarily correspond to the minimum in the \chisq plot, but is within $1\sigma$ of 1 or the minimum (except the Endgame which is closer to $2\sigma$)\footnote{Monte-Carlo tests with various random seeds showed this minimum fluctuating above and below 1. The distance from 1 therefore is not a good measure for the uncertainty in the pileup amplitude compared to the width of the \chisq parabola fit.}.


\begin{figure}[]
\centering
    \begin{subfigure}[t]{0.45\textwidth}
        \centering
        \includegraphics[width=\textwidth]{FullRatio_Chi2_Vs_PileupMultiplier_Canv_9d}
        \caption{\chisq versus pileup multiplier. The parabolic fit equation used was $y = p_{2}(x - p_{1})^{2} + p_{0}.$}
    \end{subfigure}% %you need this % here to add spacing between subfigures
    \hspace{1cm}
    \begin{subfigure}[t]{0.45\textwidth}
        \centering
        \includegraphics[width=\textwidth]{FullRatio_R_Vs_PileupMultiplier_Canv_9d}
        \caption{$R$ versus pileup multiplier. The parameter $p_{1}$ gives the sensitivity of $R$ to the value of the pileup multiplier, with units in ppm.}
    \end{subfigure}
\caption[Pileup multiplier scan]{Pileup multiplier scan. Data from 9d dataset.}
\label{fig:PMscan}
\end{figure}


\begin{table}[]
\centering
% \small
\setlength\tabcolsep{10pt}
\renewcommand{\arraystretch}{1.2}
\begin{tabular*}{0.65\linewidth}{@{\extracolsep{\fill}}lcccK}
% \begin{tabular}{@{\extracolsep{\fill}}lcccK}
  \hline
    \multicolumn{5}{c}{\textbf{Systematic Error due to Pileup Amplitude}} \\
  \hline\hline
    Dataset & \multicolumn{1}{c}{$dR/dP_{m}$} & $\sigma_{P_{m}}$ & $P_{m_{\text{min}}}$ & \multicolumn{1}{c}{$\boldsymbol{\delta R}$} \\
  \hline
    60h & $-419.3$ & 0.053 & 0.993 & 22.2 \\
    HighKick & $-372.8$ & 0.051 & 0.997 & 19.0 \\
    9d & $-245.2$ & 0.037 & 1.020 & 9.0 \\ 
    Endgame & $-335.3$ & 0.028 & 0.985 & 9.4 \\
  \hline
\end{tabular*}
\caption[Systematic error due to pileup amplitude]{Systematic error due to the pileup amplitude in the Ratio Method fits for the Run~1 precession frequency analysis datasets. The bold column gives the systematic error on \R. Units for $dR/dP_{m}$ and $\delta R$ are in ppb.}
\label{tab:systematicError_pileupMultplier}
\end{table}


The pileup phase error is the error on $R$ assuming the phase of the pileup was incorrectly constructed. This is separated into two parts. The first part is calculated by applying time-shifts to $t_{\text{doublet}}$ as given in \equref{eq:tdoublet}. Doing this artificially applies a phase shift to the pileup time spectrum. The data is then re-fit with the different pileup spectra and the change in $R$ is calculated. \figref{fig:PTSscan} shows the study results for the 9d dataset with time-shifts applied between \ns{-10} and \ns{10} in steps of \ns{1}. A clear sensitivity of $R$ to the value of time-shift is observed, however there is no clear minimum in the \chisq results. Because the width of the \chisq parabolic fit cannot be taken as the uncertainty in the pileup time-shift parameter , instead the uncertainty is taken conservatively at half the ADT at \ns{2.5}. The systematic error is calculated in the same way as for the pileup amplitude uncertainty,
    \begin{align}
        \delta R = \sigma_{P_{t}} \times \frac{dR}{dP_{t}},
    \end{align}
where $P_{t}$ is the value of the pileup time-shift. The sensitivities of $R$ to the pileup time-shift and corresponding systematic errors for the Run~1 precession frequency analysis datasets are given in \tabref{tab:systematicError_pileupTimeShift}.


\begin{figure}[]
\centering
    \begin{subfigure}[t]{0.45\textwidth}
        \centering
        \includegraphics[width=\textwidth]{FullRatio_Chi2_Vs_PileupTimeShift_Canv_9d}
        \caption{\chisq versus pileup time-shift. There is no clear minimum in the plot.} 
    \end{subfigure}% %you need this % here to add spacing between subfigures
    \hspace{1cm}
    \begin{subfigure}[t]{0.45\textwidth}
        \centering
        \includegraphics[width=\textwidth]{FullRatio_R_Vs_PileupTimeShift_Canv_9d}
        \caption{$R$ versus pileup time-shift. The parameter $p_{1}$ gives the sensitivity of $R$ to the value of the pile time-shift, with units in ppm/ns.}
    \end{subfigure}
\caption[Pileup time-shift scan]{Scan over pileup time-shift. Data from 9d dataset.}
\label{fig:PTSscan}
\end{figure}


\begin{table}[]
\centering
% \small
\setlength\tabcolsep{20pt}
\renewcommand{\arraystretch}{1.2}
\begin{tabular*}{0.7\linewidth}{@{\extracolsep{\fill}}lcK}
% \begin{tabular}{@{\extracolsep{\fill}}lcccK}
  \hline
    \multicolumn{3}{c}{\textbf{Systematic Error due to Pileup Time Shift}} \\
  \hline\hline
    Dataset & \multicolumn{1}{c}{$dR/dP_{t}$} & \multicolumn{1}{c}{$\boldsymbol{\delta R}$} \\
  \hline
    60h & 7.0 & 17.6 \\
    HighKick & 7.6 & 19.0 \\
    9d & 6.8 & 17.1 \\ 
    Endgame & 5.7 & 14.3 \\
  \hline
\end{tabular*}
\caption[Systematic error due to pilep time shift]{Systematic error due to the pileup time-shift parameter $P_{t}$ in the Ratio Method fits for the Run~1 precession frequency analysis datasets. The bold column gives the systematic error on \R. Units for $dR/dP_{t}$ and $\delta R$ are in ppb/ns and ppb respectively. The error on the $P_{t}$ is by default taken to be \ns{2.5} as described in the text. \textbf{fix spacing of table}}
\label{tab:systematicError_pileupTimeShift}
\end{table}


% -maybe put in fit start time scan of various time shifts converging to one
% -maybe add plot of nice energy threshold where pileup goes smooth and phase error would disappear


The second part of the pileup phase error comes from the choice of constant $C$ in the calculation of $E_{\text{doublet}}$ as given in \equref{eq:Edoublet}. If the energy of the pileup pulses are systematically misconstructed, then pileup shadow doublets will be added or lost near the applied energy threshold when constructing the pileup spectrum. This leads to an error on the pileup phase since it is energy-dependent. In order to calculate the systematic error from the energy construction, the parameter $C$ was scanned over from 0.9 to 1.1, in steps of 0.01. The results of the study for the 9d dataset are shown in \figref{fig:PEscan}. The systematic error on $R$ is calculated as 
    \begin{align}
        \delta R = \sigma_{C} \times \frac{dR}{dC}.
    \end{align}
Similarly to the pileup amplitude error, there is a clear minimum in the \chisq results which can be used to estimate the uncertainty in the pileup energy scale\footnote{The trend isn't as clean as in the scan over the pileup amplitude multiplier, but that is acceptable.}. \tabref{tab:systematicError_pileupC} gives the sensitivities of $R$ to the pileup energy scale, uncertainties in the pileup energy scale, and the corresponding final systematic errors for the Run~1 precession frequency analysis datasets. As shown the uncertainties on the pileup energy scale are of order 1 to 2\%, and the value for $C$ which produces the minimum in the \chisq results is consistent with 1\footnote{Because the spatial separation is turned off in the clustering portion of the reconstruction, a value of $C = 1$ is to be expected.}. Interestingly, the sensitivity of $R$ to $C$ in the 60h dataset is noticably larger than in the rest of the datasets, though the origin of this is currently unknown. \textbf{what to do about this?}


\begin{figure}[]
\centering
    \begin{subfigure}[t]{0.45\textwidth}
        \centering
        \includegraphics[width=\textwidth]{FullRatio_Chi2_Vs_PileupEnergyScaling_Canv_9d}
        \caption{\chisq versus pileup energy scale. The parabolic fit equation used was $y = p_{2}(x - p_{1})^{2} + p_{0}.$}
    \end{subfigure}% %you need this % here to add spacing between subfigures
    \hspace{1cm}
    \begin{subfigure}[t]{0.45\textwidth}
        \centering
        \includegraphics[width=\textwidth]{FullRatio_R_Vs_PileupEnergyScaling_Canv_9d}
        \caption{$R$ versus $C$. The parameter $p_{1}$ gives the sensitivity of $R$ to the value of $C$, with units in ppm.}
    \end{subfigure}
\caption[Pileup energy scale scan]{Scan over pileup energy scale. Data from 9d dataset.}
\label{fig:PEscan}
\end{figure}


\begin{table}[]
\centering
% \small
\setlength\tabcolsep{12pt}
\renewcommand{\arraystretch}{1.2}
\begin{tabular*}{0.65\linewidth}{@{\extracolsep{\fill}}lcccK}
% \begin{tabular}{@{\extracolsep{\fill}}lcccK}
  \hline
    \multicolumn{5}{c}{\textbf{Systematic Error due to Pileup Energy Scale}} \\
  \hline\hline
    Dataset & \multicolumn{1}{c}{$dR/dC$} & $\sigma_{C}$ & $C_{\text{min}}$ & \multicolumn{1}{c}{$\boldsymbol{\delta R}$} \\
  \hline
    60h & $-835.1$ & 0.023 & 0.997 & 19.4 \\
    HighKick & $-167.7$ & 0.022 & 0.995 & 3.7 \\
    9d & $-332.0$ & 0.016 & 1.000 & 5.5 \\ 
    Endgame & $-431.4$ & 0.012 & 0.982 & 5.3 \\
  \hline
\end{tabular*}
\caption[Systematic error due to fixed pileup energy scale factor]{Systematic error due to the fixed pileup energy scale parameter $C$ in the Ratio Method fits for the Run~1 precession frequency analysis. The bold column gives the systematic error on \R. Units for $dR/dC$ and $\delta R$ are in ppb.}
\label{tab:systematicError_pileupC}
\end{table}


\tabref{tab:PileupErrorsTotal} gives the quadrature sum for the total pileup systematic errors for the Run~1 precession frequency analysis datasets. As shown for each dataset the total error is below the target final error of \SI{40}{ppb} in spite of the contamination in the pileup shadow method. For future runs of the experiment with increased rate and therefore increased pileup, these errors may grow. In that case either the pileup shadow method might need to be improved to account for the contamination and pileup triplets, or discarded in favor of a different method.


\begin{table}[]
\centering
\setlength\tabcolsep{10pt}
\renewcommand{\arraystretch}{1.2}
\begin{tabular*}{\linewidth}{@{\extracolsep{\fill}}lcGGGG}
  \hline
    \multicolumn{6}{c}{\textbf{Total Pileup Systematic Errors}} \\
  \hline\hline
    Type of Error & Parameter & \multicolumn{1}{c}{60h} & \multicolumn{1}{c}{HighKick} & \multicolumn{1}{c}{9d} & \multicolumn{1}{c}{Endgame} \\
  \hline
    Amplitude & $P_{m}$  & 22.2 & 19.0 & 9.0  & 9.4 \\
    Phase     & $P_{t}$  & 17.6 & 19.0 & 17.1 & 14.3 \\
    Phase     & $C$      & 19.4 & 3.7  & 5.5  & 5.3 \\
  \hline
    Quadrature sum &  & 34.3 & 27.1 & 20.1 & 17.9 \\
  \hline 
\end{tabular*}
\caption[Total pileup systematic errors]{Total pileup systematic errors for the Run~1 precession frequency analysis datasets.}
\label{tab:PileupErrorsTotal}
\end{table}




\subsection{Gain systematic errors}
\label{sub:gainerror}


-talk about the equations here - slight reference to either detector section or reconstruction section in this chapter - might want to put this after pileup if I introduce various equations and the like in that section


\subsection{CBO systematic errors}
\label{sub:cboerror}



\subsection{Lost muon systematic errors}
\label{sub:lostmuonserror}


As mentioned in \secref{subsec:lostmuons}, the triples spectrum is made with cuts as defined in \tabref{tab:lostmuoncuts}. Various backgrounds are subtracted off the triples spectrum in order to generate a clean sample of lost muons. \tabref{tab:lostmuonsvariousfits} gives the change in \R for various sets of cuts and background subtractions for the 9d and Endgame datasets. As shown the various different backgrounds and cuts ultimately make very little difference in the final fitted \R value. Similarly, stable beam contaminants in the form of deuterons and protons contaminate the lost muon spectrum. The former are largely removed by straightforward \DT and $\Delta t_{13}$ cuts. The latter can be mostly removed by cutting on the negative side of the $\Delta t_{13}$ distribution which separates the populations more readily, \figref{fig:deltaT13}, with $\Delta t_{13} \leq \SI{12.5}{ns}$. While this does largely remove the protons at the cost of statistics, the fitted \K parameter simply grows larger to compensate. Ultimately the effect on \R is negligible, with $\Delta R = \SI{-0.3}{ppb}$. The sum of these separate types of errors is conservatively taken at \SI{1}{ppb} for all datasets\footnote{For the T-Method fits, though the changes in \R are noticably larger with the various cuts, they are still the same order of magnitude and the error is conservatively below \SI{1}{ppb}.}.


\begin{table}[]
\centering
\setlength\tabcolsep{10pt}
\renewcommand{\arraystretch}{1.2}
\begin{tabular*}{1\linewidth}{@{\extracolsep{\fill}}lcccc}
  \hline
    \multicolumn{5}{c}{\textbf{$\Delta R$ with Various Lost Muon Cuts}} \\
  \hline\hline
    & \multicolumn{2}{c}{9d Dataset} & \multicolumn{2}{c}{Endgame Dataset} \\
  \hline
    Type of fit or cut & $\Delta R$ (ppb) & \K & $\Delta R$ (ppb) & \K \\
  \hline
    No quadruple subraction & 0.2 & 1.811 & -0.1 & 1.717 \\
    No accidental subraction & 0.1 & 2.503 & $<0.1$ & 2.339 \\
    $\Delta t_{13} \leq \SI{12.5}{ns}$ & 0.1 & 4.469 & -0.3 & 4.248 \\
    $\SI{5}{ns} \leq \Delta t_{12, 23} \leq \SI{8.5}{ns}$ & N/A & N/A & $<0.1$ & 1.709 \\
    $\SI{100}{\MeV} \leq E_{1,2,3} \leq \SI{500}{\MeV}$ & N/A & N/A & $<0.1$ & 2.09 \\
  \hline 
\end{tabular*}
\caption[Effect on fitted R value due to lost muon cuts in the 9d and Endgame datasets]{Effect on the fitted \R value in the 9d and Endgame datasets with various different cuts used or backgrounds subtracted. Ultimately how the muon loss spectrum $L(t)$ is created has little bearing on the final fitted \R value. Also included are the various corresponding \K values which compensate for the level of statistics contained within $L(t)$ due to the various cuts. \textbf{decide whether to fill out the N/A 9d fits or not}}
\label{tab:lostmuonsvariousfits}
\end{table}


\begin{figure}[]
    \centering
    \includegraphics[width=.8\textwidth]{deltaT13_timeInFill_noCuts_Endgame}
    \caption[Lost muon $\Delta t_{13}$ distribution as a function of time in-fill]{Lost muon $\Delta t_{13}$ distribution as a function of time in-fill.}
    \label{fig:deltaT13}
\end{figure}



Of more interest is the choice of fixed value for the \K parameter in the ratio fits, fixed from the corresponding T-Method fits. The systematic error can be determined by simply scanning over the value of \K as in \figref{fig:kappaLossScan}, where the error on the parameter is that taken from a T-Method fit to the same data\footnote{The size of the error on \K is entirely due to statistics.}. \tabref{tab:systematicError_kappaLoss} gives the systematic errors on \R for the Run~1 precession frequency analysis datasets. Interestingly enough, though the Endgame dataset has the most lost muons, it is seen that the change in \R versus \K is larger in the other datasets. This is potentially due to the fact that \textbf{try to come up with something for this unless I can explain it away as an oddity.}


\begin{figure}[]
    \centering
    \includegraphics[width=.7\textwidth]{FullRatio_R_Vs_kappa_loss_Canv_9d}
    \caption[Scan over fixed \K]{The sensitivity of \R to the fixed \K parameter. Error bars have been removed from the plot. Units are in ppm, data from the 9d dataset.}
    \label{fig:kappaLossScan}
\end{figure}


\begin{table}[]
\centering
% \small
% \setlength\tabcolsep{10pt}
\renewcommand{\arraystretch}{1.2}
\begin{tabular*}{0.75\linewidth}{@{\extracolsep{\fill}}lGcJG}
  \hline
    \multicolumn{5}{c}{\textbf{Systematic Error due to Fixed $\kappa_{loss}$}} \\
  \hline\hline
    Dataset & \multicolumn{1}{c}{$dR/d\kappa_{loss}$} & $\sigma_{\kappa_{loss}}$ & \multicolumn{1}{c}{$\boldsymbol{\delta R}$} & \multicolumn{1}{c}{$\Delta R$ (with - without)} \\
  \hline
    60h & -3.5 & 0.338 & 1.2 & -31.4 \\
    HighKick & -7.1 & 0.697 & 4.9 & -40.1 \\
    9d & -18.3 & 0.170 & 3.1 & -45.7 \\
    Endgame & -2.6 & 0.038 & 0.1 & -6.1 \\
  \hline
\end{tabular*}
\caption[Systematic error due to fixed $\kappa_{loss}$]{Systematic error due to the fixed $\kappa_{loss}$ parameter in the Ratio Method fits for the Run~1 precession frequency analysis datasets. The bold column gives the systematic error on \R. The last column on the right gives the change in \R with the lost muons term included versus without, providing an absolute upper bound on systematic error. All units are in ppb except for the $\sigma_{\kappa_{loss}}$ parameter which is unit-less and comes from the T-Method fits.}
\label{tab:systematicError_kappaLoss}
\end{table}




- a systematic error comes from the fact that muons are lost preferentially at earlier times (changing over the course of a fill)
-if lost muons originate from different points in the beam line, then they will have precessed a different amount leading to a different phase


-this error dominates the uncertainty...


-probably move losses fraction figure to here somewhere


The fractional losses (the integral in \equref{eq:lambdalosses} times the final fitted $\kappa_{loss}$ parameter) for the Run~1 precession frequency analysis datasets are shown in \figref{fig:fractionallosses}.



\begin{figure}[]
    \centering
    \includegraphics[width=.8\textwidth]{fractionalLosses_dataset_comparison}
    \caption[Fractional muon losses in the analyzed Run~1 datasets]{Fractional losses for the Run~1 precession frequency analysis datasets. The curves begin at \mus{30.2} which is where the fit begins. A value of 1\% at a specific time $t$ indicates that there are 1\% fewer stored muons at that time than there would be if there were no losses at all. The Endgame and 60h datasets can be seen to have the most losses, while the 9d and HighKick have less. This is due to the higher kicks in the latter datasets which put the muon beam on a more central orbit. The upward tail at the end of the Endgame dataset corresponds to the remnant proton contamination.}
    \label{fig:fractionallosses}
\end{figure}






-also Sudeshna's talk in the Elba collaboration meeting - this probably just for the systematic error






-put in a chart at the end summarizing the total lost muons systematic error - unless I want to keep the pieces separate, and have one giant chart at the end which might be too much



\subsection{Ratio construction systematic errors}
\label{sub:TimeShiftingParameters}

-point out how slopes vary positively and negatively and then claim no real systematic error


\begin{figure}[]
\centering
    \begin{subfigure}[t]{0.45\textwidth}
        \centering
        \includegraphics[width=\textwidth]{FullRatio_R_Vs_gm2PeriodGuess_Canv_60h}
        \caption{}
    \end{subfigure}% %you need this % here to add spacing between subfigures
    \hspace{1cm}
    \begin{subfigure}[t]{0.45\textwidth}
        \centering
        \includegraphics[width=\textwidth]{FullRatio_R_Vs_weightingLifetime_Canv_60h}
        \caption{}
    \end{subfigure}
\caption[]{}
\label{fig:}
\end{figure}


\begin{table}[]
\centering
% \small
\setlength\tabcolsep{20pt}
\renewcommand{\arraystretch}{1.2}
\begin{tabular*}{0.7\linewidth}{@{\extracolsep{\fill}}lHH}
% \begin{tabular}{@{\extracolsep{\fill}}lcccK}
  \hline
    \multicolumn{3}{c}{\textbf{Sensitivity to Ratio Construction Parameters}} \\
  \hline\hline
    Dataset & \multicolumn{1}{c}{$dR/d_{T_{a}}$} & \multicolumn{1}{c}{$dR/d_{\tau_{\mu}}$} \\
  \hline
    60h & 0.1 & 6.9 \\
    HighKick & -0.1 & -4.1 \\
    9d & <0.1 & -1.1 \\ 
    Endgame & <0.1 & 0.6 \\
  \hline
\end{tabular*}
\caption[]{units in ppb per... \textbf{fix spacing of table}}
\label{tab:}
\end{table}






\subsection{Binning systematic errors}


-for bin edge it was verified that a shift of one bin width returned the same fit results as a shift of 0, that with the negligible slope implies no systematic effect on R

-for the bin width not only do the slopes very widely when scanning over the different datasets, but upon inspection of the points in the plot there is no clear trend, that combined with the results from blah analysis showing that the optimal bin width is 149.2 ns implies there is not systematic effect on R - if one wanted to be very conservative they could take the uncertainty in the bin width as X ns (probably something like 0.1), and the systematic errors would be absolutely negligible


\begin{figure}[]
\centering
    \begin{subfigure}[t]{0.45\textwidth}
        \centering
        \includegraphics[width=\textwidth]{FullRatio_R_Vs_binWidth_Canv_9d}
        \caption{}
    \end{subfigure}% %you need this % here to add spacing between subfigures
    \hspace{1cm}
    \begin{subfigure}[t]{0.45\textwidth}
        \centering
        \includegraphics[width=\textwidth]{FullRatio_R_Vs_binEdgeShift_Canv_9d}
        \caption{}
    \end{subfigure}
\caption[]{}
\label{fig:}
\end{figure}


\begin{table}[]
\centering
% \small
\setlength\tabcolsep{20pt}
\renewcommand{\arraystretch}{1.2}
\begin{tabular*}{0.7\linewidth}{@{\extracolsep{\fill}}lGG}
% \begin{tabular}{@{\extracolsep{\fill}}lcccK}
  \hline
    \multicolumn{3}{c}{\textbf{Sensitivity to Binning Parameters}} \\
  \hline\hline
    Dataset & \multicolumn{1}{c}{$dR/d_{\text{bin width}}$} & \multicolumn{1}{c}{$dR/d_{\text{bin edge}}$} \\
  \hline
    60h & 24.5 & -0.1 \\
    HighKick & 6.0 & -0.7 \\
    9d & -22.8 & -0.3 \\ 
    Endgame & -41.7 & -0.6 \\
  \hline
\end{tabular*}
\caption[]{units in ppb/ns \textbf{fix spacing of table}}
\label{tab:}
\end{table}



\subsection{Random seed systematic errors}

-not a systematic error per se but ... 

\subsection{Systematic error summary}

-include here a big table (with each of the individual errors making up say pileup and the like) of all systematic errors



