\graphicspath{ {Body/Figures/ExperimentalOverview/Decay/} }

\chapter{Muon g-2 at Fermilab, E989}
\label{chapter:Muon g-2 at Fermilab, E989}
\thispagestyle{myheadings} % should I be including this at the top of every section page??

\section{Principle Technique}
\label{sec:PrincipleTechnique}

In a magnetic field, particles will orbit at the cyclotron frequency 
        \begin{align} \label{eq:wc}
        	\omega_{c} = -\frac{Qe}{\gamma m}B,
        \end{align}
and their spins will turn at the precession frequency
        \begin{align} \label{eq:ws}
        	\omega_{s} = -g\frac{Qe}{2m}B - (1-\gamma)\frac{Qe}{\gamma m}B,
        \end{align}
where $Q = \pm 1$ and $e > 0$. The difference between these two frequencies gives
        \begin{align} \label{eq:wasimple}
        	\omega_{a} = \omega_{s} - \omega_{c} = -\frac{g-2}{2}\frac{Qe}{m}B = - a_{\mu} \frac{Qe}{m}B,
        \end{align}
a measureable frequency that is directly proportional to the property of significance, \amu. By measuring the spin difference frequency for muons and the magnetic field, \amu can be determined. In the presence of an electric field, which is necessary to store the muon beam, this expands to 
        \begin{align} \label{eq:waelectric}
            \vec{\omega}_{a} = -\frac{Qe}{m} [a_{\mu}\vec{B} - (a_{\mu} - \frac{1}{\gamma^{2}-1})(\vec{\beta} \times \vec{E}) ],
        \end{align}
where now the measurable quanties are vector quantities. Finally, for realistic cases of muon momentum is non-orthogonal to the magnetic field, the spin difference frequency becomes
        \begin{align} \label{eq:wafinal}
            \vec{\omega}_{a} = -\frac{Qe}{m} [a_{\mu}\vec{B} - a_{\mu} (\frac{\gamma}{\gamma+1})(\vec{\beta} \cdot \vec{B})\vec{B} - (a_{\mu} - \frac{1}{\gamma^{2}-1})(\vec{\beta} \times \vec{E}) ].
        \end{align}
If the motion of the muons is largely perpendicular to the magnetic field, then the second term is small and can be corrected for. If the particles have a momentum of approximately 3.09 GeV/c, the so called ``magic momentum,'' then the third term is small and can be corrected for. 

In order to measure the spin difference frequency of the muons, a clever technique is used. Decay muons in the pion rest frame are 100\% polarized due to the conservation of angular momentum and the fact that the decay neutrino must have a specific helicity. Within a pion beam then the highest and lowest energy decay muons are polarized. Muons will 
decay to positrons with a lifetime of about 2.2 $\mu$s, and the positrons with the highest energies will be correlated with the muon spin, a so called ``self-analyzing'' decay. The single available decay state for a maximum energy positron illustrates this in Figure \ref{fig:MuonDecay}.

\begin{figure}[]
	\caption[Muon Decay - Max Energy Positron]{Make a new version of this picture, and improve this caption.: The single available decay state for maximum energy decay positrons. Due to the conservation of angular momentum and the single possible helicity states of the decay neutrino and anti-neutrino, the spin of the decay positron is exactly equal to the spin of the muon at the time of the decay.}
	\centering
	\includegraphics[width=0.9\textwidth]{MuonDecay}
	\label{fig:MuonDecay}
\end{figure}






-explain the physics
-explain how we get at the physics with our ring and detectors


don't measure all decay positrons


By injecting a large ensemble of muons and 
by measuring a subset of ensemble of muons....


Careful with spin vs polarization









\section{Detector Systems}
\label{sec:DetectorSystems}


\subsection{Calorimeters}
\label{sec:Calorimeters}

% what they do
% what they are
% how they work


Electromagnetic calorimeters measure the times and energies of decay positrons as they curl inward from the storage region. There are 24 calorimeters located symmetrically around the inside of the ring in close proximity to the vacuum chamber, as shown in Figure \ref{fig:}. They lie close to the storage region in order to measure a large fraction of the total number of observable decay positrons, including the high energy decay positrons which curl inward only slightly more than the muons themselves do. Because they are in close proximity to the storage region and by extension the magnetic field, the calorimeters must be non-magnetic in order to avoid perturbing the magnetic field. Each calorimeter consists of 54 channels of PbF\textsubscript{2} crystals arrayed in a 6 high by 9 wide array, which measure Cerenkov light emitted by the impinging positrons as they pass through the crystals \cite{Fienberg:2014kka}. (Picture of single calo and its crystals here.) Each crystal is 2.5 x 2.5 x 14 cm\textsuperscript{3}. The light is read out by large area silicon photo-multiplier (SiPM) sensors.


In order to determine \amu to the precision goal, there are modest requirements on the performance of the calorimeters. They must have a relative energy resolution of better than 5\% at 2 GeV, in order for proper event selection \cite{TDR}. They must have a timing resolution of better than 100 ps. The calorimeters must be able to resolve multiple incoming hits through temporal or spatial separation at 100\% efficiency for time separations of greater than 5 ns in order to reduce the pileup systematic error due to the high rate. Finally, the gain of the measured hits must be stable to $<$ 0.1\% over a 200 $\mu$s time period within a fill, and unaffected by a pulse arriving in the same channel a few nanoseconds earlier. The long term gain stability over a time period of order seconds must be $<$ 1\%.

(I've condensed quite a bit this section from the TDR - is that okay?)


To satisy these requirements SiPM sensors were chosen over PMTS...



and is wrapped in black Tedlar\textregistered\ foil



\subsection{Laser System}
\label{LaserSystem}
% should make this it's own section

For the gain, there is a laser system...




% magnetic nature of calorimeters

% energy resolution
% gain
% high rate
% acceptance

% design of the vacuum chambers in order to...

% picture of calo placement
% picture of actual calo station


\cite{Kaspar:2016ofv}



\subsection{Template Fitting}
\label{sec:TemplateFitting}


