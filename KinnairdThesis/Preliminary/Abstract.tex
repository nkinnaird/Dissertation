%!TEX root = ../thesis.tex

% ABSTRACT

This abstract was copied from my departmental seminar, and it needs to be updated for my thesis: One of the few indications for new physics is the discrepancy between the theoretical and experimental values for the anomalous magnetic moment of the muon. There is a 3 to 4 sigma discrepancy between theory and the last experimental measurement held at Brookhaven National Laboratory in 2001, which measured the muon g-2 to 540 parts per billion. This discrepancy has been consistent for many years now with ever improving theoretical calculations and other experimental measurements. In order to resolve or confirm this difference, a new experiment is underway at Fermilab to measure the muon g-2 to 4 times higher precision at 140 ppb. Muon g-2 at Fermilab gathered its first production data in 2018, and is currently taking data now. I will describe the principles of the experiment and detail two specific parts of the analysis that I have been involved in. These include track fitting and precession frequency analysis of the Run 1 data.
