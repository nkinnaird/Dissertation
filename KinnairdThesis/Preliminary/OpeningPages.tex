% This file contains all the necessary setup and commands to create
% the preliminary pages according to the buthesis.sty option.

\title{MEASUREMENT OF THE ANOMALOUS MAGNETIC MOMENT OF THE POSITIVE MUON TO .SOMETHING PARTS PER BILLION}

\author{Nicholas Brennan Kinnaird}

\prevdegrees{B.S., University of Texas at Austin, 2013\\
	B.S., University of Texas at Austin, 2013\\
    M.A., Boston University, 2016}

\department{Graduate School of Arts and Sciences}

% Degree year is the year the diploma is expected, and defense year is
% the year the dissertation is written up and defended. Often, these
% will be the same, except for January graduation, when your defense
% will be in the fall of year X, and your graduation will be in
% January of year X+1
\defenseyear{2019}
\degreeyear{2019}

% For each reader, specify appropriate label {First, Second, Third},
% then name, and title. IMPORTANT: The title should be:
%   "Professor of Electrical and Computer Engineering",
% or similar, but it MUST NOT be:
%   Professor, Department of Electrical and Computer Engineering"
% or you will be asked to reprint and get new signatures.
% Warning: If you have more than five readers you are out of luck,
% because it will overflow to a new page. You may try to put part of
% the title in with the name.
\reader{First}{B. L. Roberts, PhD}{Professor of Physics}
\reader{Second}{R. M. Carey, PhD}{Professor of Physics}
\reader{Third}{J. P. Miller, PhD}{Professor of Physics}

% The Major Professor is the same as the first reader, but must be
% specified again for the abstract page.
\majorprof{B. L. Roberts}{{Professor of Physics}}


%%%%%%%%%%%%%%%%%%%%%%%%%%%%%%%%%%%%%%%%%%%%%%%%%%%%%%%%%%%%%%%%  

%                       PRELIMINARY PAGES
% According to the BU guide the preliminary pages consist of:
% title, copyright (optional), approval,  acknowledgments (opt.),
% abstract, preface (opt.), Table of contents, List of tables (if
% any), List of illustrations (if any). The \tableofcontents,
% \listoffigures, and \listoftables commands can be used in the
% appropriate places. For other things like preface, do it manually
% with something like \newpage\section*{Preface}.

% This is an additional page to print a boxed-in title, author name and
% degree statement so that they are visible through the opening in BU
% covers used for reports. This makes a nicely bound copy. Uncomment only
% if you are printing a hardcopy for such covers. Leave commented out
% when producing PDF for library submission.
%\buecethesistitleboxpage

% Make the titlepage based on the above information.  If you need
% something special and can't use the standard form, you can specify
% the exact text of the titlepage yourself.  Put it in a titlepage
% environment and leave blank lines where you want vertical space.
% The spaces will be adjusted to fill the entire page.
\maketitle
\cleardoublepage

% The copyright page is blank except for the notice at the bottom. You
% must provide your name in capitals.
\copyrightpage
\cleardoublepage

% Now include the approval page based on the readers information
\approvalpage
\cleardoublepage

% Otional dedication page should go here. Page numbers start here at iv. How to move this down and get page numbers working?
\newpage
\thispagestyle{empty}
\vspace*{1in}
\section*{\centerline{Dedication}}
%!TEX root = ../thesis.tex

I dedicate this thesis to my parents, John, Nett, and Zindy. Thank you for all support and encouragement. I love you all dearly.

\cleardoublepage

% The acknowledgment page should go here. Use something like
% \newpage\section*{Acknowledgments} followed by your text.
\newpage
\section*{\centerline{Acknowledgments}}
Here go all your acknowledgments. You know, your advisor, funding agency, lab
mates, etc., and of course your family.

As for me, I would like to thank Jonathan Polimeni for cleaning up old LaTeX
style files and templates so that Engineering students would not have to suffer
typesetting dissertations in MS Word. Also, I would like to thank IDS/ISS
group (ECE) and CV/CNS lab graduates for their contributions and tweaks to this
scheme over the years (after many frustrations when preparing their final
document for BU library). In particular, I would like to thank Limor Martin who
has helped with the transition to PDF-only dissertation format (no more printing
hardcopies -- hooray !!!)

The stylistic and aesthetic conventions implemented in this LaTeX
thesis/dissertation format would not have been possible without the help from
Brendan McDermot of Mugar library and Martha Wellman of CAS.

Finally, credit is due to Stephen Gildea for the MIT style file off which this
current version is based, and Paolo Gaudiano for porting the MIT style to one
compatible with BU requirements.

\vskip 1in

\noindent
Janusz Konrad\\
Professor\\
ECE Department
\cleardoublepage

% The abstractpage environment sets up everything on the page except
% the text itself.  The title and other header material are put at the
% top of the page, and the supervisors are listed at the bottom.  A
% new page is begun both before and after.  Of course, an abstract may
% be more than one page itself.  If you need more control over the
% format of the page, you can use the abstract environment, which puts
% the word "Abstract" at the beginning and single spaces its text.

\begin{abstractpage}
%!TEX root = ../thesis.tex

% ABSTRACT


One of the few indications for new physics is the discrepancy between the theoretical and experimental values for the anomalous magnetic moment of the muon. There is a discrepancy of 3 to 4 standard deviations between theory and the last experimental measurement made at Brookhaven National Laboratory in 2001, which measured the muon magnetic anomaly \amu to 540 parts per billion (ppb). This discrepancy has been consistent for many years with ever improving theoretical calculations. In order to resolve or confirm this discrepancy experiment E989, Muon \gmtwo, is underway to measure \amu to 4 times higher precision at \SI{140}{ppb}. In Run~1 E989 gathered its first production data, consisting of approximately \SI{8e9}{} decay positrons above an energy threshold of \SI{1.7}{\GeV}.

This dissertation describes the experimental measurement, the detectors, the precession frequency extraction, and the track fitting of decay positrons in Run~1. The track fitting is done using a \chisq minimization algorithm to fit tracks propagated within a Geant4 reconstruction simulation including error propagation. The precession frequency is extracted using an analysis technique called the Ratio Method. The Ratio Method takes the ratio of time-shifted decay positron spectra in order to remove the decay exponential along with slowly varying effects in the data. Precession frequency extraction analyses for four near-final Run~1 datasets are presented with full systematic error evaluations. The total Run~1 precession frequency error determined in this analysis is $\SI{491.2}{ppb}$, where the error is statistics dominated. Combined with the expected error in the magnetic field measurement of \SI{140}{ppb}, the expected final error on \amu for Run~1 of E989 is $\mathcal{O}(\SI{510}{ppb})$, comparable to the previous measurement.







\end{abstractpage}
\cleardoublepage

% overwrite link color for intro pages, which is red in the main text (should it actually be black there too?)
{\hypersetup{linkcolor=black}

% Table of contents comes after preface
\tableofcontents
\cleardoublepage

% If you do not have tables, comment out the following lines
\newpage
\listoftables
\cleardoublepage

% If you have figures, uncomment the following line
\newpage
\listoffigures
\cleardoublepage

% List of Abbrevs is NOT optional (Martha Wellman likes all abbrevs listed)
\chapter*{List of Abbreviations}
\begin{center}
  \begin{tabular}{lll}
    \hspace*{2em} & \hspace*{1in} & \hspace*{4.5in} \\
    CAD  & \dotfill & Computer-Aided Design \\
    CO   & \dotfill & Cytochrome Oxidase \\
    DOG  & \dotfill & Difference Of Gaussian (distributions) \\
    FWHM & \dotfill & Full-Width at Half Maximum \\
    LGN  & \dotfill & Lateral Geniculate Nucleus \\
    ODC  & \dotfill & Ocular Dominance Column \\
    PDF  & \dotfill & Probability Distribution Function \\
    $\mathbb{R}^{2}$  & \dotfill & the Real plane \\
  \end{tabular}
\end{center}
\cleardoublepage

} % end hypersetup override block

% END OF THE PRELIMINARY PAGES

\newpage
\endofprelim
