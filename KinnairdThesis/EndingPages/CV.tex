%!TEX root = ../thesis.tex


\thispagestyle{myheadings}

% original template from https://www.overleaf.com/latex/templates/simple-resume/gtrrjrsrfkvk


% \usepackage{latexsym}
% \usepackage[empty]{fullpage}
% \usepackage{titlesec}
% \usepackage{marvosym}
% \usepackage[usenames,dvipsnames]{color}
% \usepackage{verbatim}
% \usepackage{enumitem}
% \usepackage[pdftex, hidelinks]{hyperref}
% \usepackage{fancyhdr}

% \usepackage[charter]{mathdesign} % Bitstream Charter
% \usepackage{newpxtext,newpxmath} % Palatino


%-------------------------
% Custom commands
\newcommand{\resumeItem}[2]{
  \item\small{
    \textbf{#1}{: #2 \vspace{-2pt}}
  }
}

\newcommand{\resumeItemNoBullet}[2]{
  \item[]\small{
    \hspace{-9pt}\textbf{#1}{: #2 \vspace{-6pt}}
  }
}

\newcommand{\resumeSubheading}[4]{
  \vspace{-1pt}\item[]
  \begin{tabular*}{0.98\textwidth}{l@{\extracolsep{\fill}}r}
      \hspace{-10pt}\textbf{#1} & #2 \\
      \hspace{-10pt}\textit{\small#3} & \textit{\small #4} \\
    \end{tabular*}\vspace{-5pt}
}

\newcommand{\resumeSubItem}[2]{\resumeItem{#1}{#2}\vspace{-4pt}}

\renewcommand{\labelitemii}{$\circ$}

\newcommand{\resumeSubHeadingListStart}{\begin{itemize}[leftmargin=*]}
\newcommand{\resumeSubHeadingListEnd}{\end{itemize}}
\newcommand{\resumeItemListStart}{\begin{itemize}}
\newcommand{\resumeItemListEnd}{\end{itemize}\vspace{-5pt}}

% custom commands
\newcommand{\shorterSection}[1]{\vspace{-10pt}\section*{#1}}

%-------------------------------------------
%%%%%%  CV STARTS HERE  %%%%%%%%%%%%%%%%%%%%%%%%%%%%


%----------HEADING-----------------
\begin{center}
  \textbf{{\Large Nicholas Kinnaird}} \\
  \small 512-922-0534 $\vert$ nickkinn@bu.edu $\vert$ 1101 Iroquois Ave Apt 1401, Naperville, IL 60563
\end{center}

%-----------EDUCATION-----------------
\shorterSection{Education}
  \resumeSubHeadingListStart
    \resumeSubheading
      {Boston University}{Boston, MA}
      {Ph.D. in Physics}{2020}{
      \resumeItemNoBullet{Dissertation}{Muon Spin Precession Frequency Extraction and Decay Positron Track Fitting in Run 1 of the Fermilab Muon \gmtwo Experiment}
    }
    \resumeSubheading
      {Boston University}{Boston, MA}
      {M.A. in Physics}{2016}{
    }
    \resumeSubheading
      {University of Texas at Austin}{Austin, TX}
      {B.S. in Physics}{2013}
    \resumeSubheading
      {University of Texas at Austin}{Austin, TX}
      {B.S. in Mathematics}{2013}{}
  \resumeSubHeadingListEnd

% %-----------SKILLS-----------------
% \shorterSection{Skills}
%   \resumeSubHeadingListStart
%   \small
%     \item{
%      \textbf{Languages}{: Python, C++, SQL, Java, Swift}
%      \hfill
%      \textbf{Technologies}{: GCP, AWS, GitHub, GitLab, Docker}
%     }
%     \vspace{-5pt}
%     \item{
%      \textbf{Libraries}{: TensorFlow, PyTorch, Keras, Scikit-Learn, Numpy, Pandas, Spark, Jupyter, OpenCV, PIL, OpenCL, OpenGL, CUDA}
%     }
% \resumeSubHeadingListEnd

% %-----------EXPERIENCE-----------------
% \shorterSection{Experience}
%   \resumeSubHeadingListStart

%     \resumeSubheading
%       {CCC Information Services}{Chicago, IL}
%       {R\&D Engineer Intern (Machine Learning)}{May 2018 - Present}
%       \resumeItemListStart
%         \resumeItem{RotNet}
%           {Designed and trained a shallow CNN to classify rotated images. Achieved F1-score of 0.99 on test set of 1M+ images}
%         \resumeItem{TagNet}
%           {A computer vision model to classify views of automobile images
%             \begin{itemize}
%                 \item Implemented an expectation maximization algorithm to sample and classify images from 6M+ unlabeled images to prepare a clean dataset of 100K+ images for training
%                 \item Designed and trained a low-complexity CNN to classify 20+ views of automobile images resulting in 30\% improvement in  F1-score compared to SVM model
%                 \item Reduced model complexity and size by 40\% by freezing model and performing post-training quantization
%             \end{itemize}
%           }
%         \resumeItem{TVR}
%           {A computer vision model to determine if image is total loss or repairable
%           \begin{itemize}
%               \item Trained ensemble of CNN architectures on 1M+ automobile images to classify vehicles as total loss or repairable resulting in 25\% higher weighted F1-score and 60\% decrease in model size compared to older iteration
%               \item Incorporated first notice of loss information using NLP to increase  F1-score by 10\%
%           \end{itemize}
%           }
%       \resumeItemListEnd

%     \resumeSubheading
%       {Reliance Communications}{Mumbai, India}
%       {Software Engineer Intern}{May 2016 - Aug 2016}
%       \resumeItemListStart
%         \resumeItem{Optimal Node Search}
%           {Implemented Dijkstra's algorithm on 10K+ network nodes to find shortest path for signal propagation resulting in 25\% reduction in costs}
%       \resumeItemListEnd

%     \resumeSubheading
%       {OSSCube}{New Delhi, India}
%       {Software Engineer Intern}{May 2015 - Aug 2015}
%       \resumeItemListStart
%         \resumeItem{Squeek Twitter iOS}
%           {Designed and developed Twitter client using Fabric SDK}
%       \resumeItemListEnd

%   \resumeSubHeadingListEnd

% %-----------PROJECTS-----------------
% \shorterSection{Projects}
%   \resumeSubHeadingListStart
%     \resumeSubItem{OCR using Conditional Random Fields}
%      {A probabilistic graphical model for sequential character recognition
%         \vspace{-5pt}
%         \begin{itemize}
%             \item Implemented a CRF in $O(m|\mathcal{Y}|^2)$ time to achieve a 84\% letter-wise accuracy on UPenn OCR dataset
%             \item Implemented OpenMPI CRF using PETSc and Tao to achieve 77.1\% letter-wise accuracy
%         \end{itemize}
%      }
%     \resumeSubItem{ARYouThereYet}{An augmented reality application developed on ARKit with dynamic AR nodes}
%     \resumeSubItem{Aspect-based Sentiment Analysis}
%       {Implemented Deep Memory Networks to achieve 78.66\% accuracy, 0.69 F1-score on SemEval 2014 dataset}
%     \resumeSubItem{Iris Speech to Code}
%       {A natural speech to code converter for aiding programmers with disabilities
%         \vspace{-5pt}
%         \begin{itemize}
%             \item Trained an intent classification model in Microsoft Luis to classify 15+ classes or commands
%             \item Implemented a message passing protocol using RabbitMQ to broker messages between Google API, ElectronJS, and VS Code
%         \end{itemize}
%       }
%     \resumeSubItem{AI Lifeguard}{Trained a 3D-CNN model on Microsoft Azure for action localization on drowning people in swimming pools. Achieved mean IOU score of 0.45}
%   \resumeSubHeadingListEnd

% %-----------Addtional Experience & Achievements-----------------
% \shorterSection{Additional Experience \& Achievements}
%   \resumeSubHeadingListStart
%   \small
%     \item{Presented poster on \textit{Tiramisu DenseNet Architecture for Precise Segmentation} for Intel AI at \textbf{CVPR 2018}}
%     \vspace{-5pt}
%     \item{Selected as an \textbf{Intel AI Student Ambassador} (only 150 students) to research, publish, and share work on machine learning and deep learning}
%     \vspace{-5pt}
%     \item{Won \textit{Best Microsoft Hack} out of 220 teams at \textbf{HackHarvard 2017}}
%     \vspace{-5pt}
%     \item{Placed 16/50 at Google Games: Campus Edition 2017 at UIC}
%     \vspace{-5pt}
%     \item{Won \textit{Best Technical Innovation} award (out of 800 students) at \textbf{Amity University Convocation 2017}}
%     \vspace{-5pt}
%     \item{Elected as a \textit{Vice-Chair} for \textbf{ACM Amity Student Chapter} out of 800 students at Amity University based on high-achieving and technically strong undergraduate students}
%   \resumeSubHeadingListEnd
%-------------------------------------------
\end{document}
