\documentclass[12pt,letterpaper]{article}

\usepackage[intlimits]{amsmath}
\usepackage{siunitx}
% \usepackage[utf8]{inputenc}
% \usepackage{amsfonts,amssymb}
\usepackage{xspace}

\def\wa{$\omega_{a}$\xspace}
\def\chisq{$\chi^{2}$\xspace}
\def\gmtwo{$g-2$\xspace}
\def\amu{$a_{\mu}$\xspace}
\def\g{$g$\xspace}


\begin{document}



\begin{center}
\Large{Ph.D. Dissertation Prospectus} \\\vspace{5mm}
\textbf{\Large{Measurement of the Anomalous Magnetic Moment of the Muon to X parts per billion in Run 1 of the Fermilab Muon \boldmath{$g-2$} Experiment}} \\\vspace{5mm}
\large{Nicholas Kinnaird} \\\vspace{5mm}
\large{Boston University Physics Department} \\\vspace{5mm}
\end{center}


%==========================================================================%
\section*{Overview}


The Fermilab Muon \gmtwo Experiment (E989) is measuring the anomalous magnetic moment of the muon \amu to high precision. The goal is 140 parts per billion, which represents a four-fold improvement over the the previous best experimental measurement of 540 parts per billion, made by the E821 collaboration at Brookhaven National Laboratory in 2001. There is currently a \SIrange{3}{4}{\sigma} discrepancy between theory and experiment, which the new measurement has been designed to resolve or confirm. The difference between theoretical and experimental values might be attributed to new physics. The new measurement is based upon the same principles as the last experiment, and if it measures the same central value, the discrepancy would be pushed over the 5$\sigma$ level needed to classify it as a discovery. The experiment is located at Fermi National Accelerator Laboratory which has the facilities necessary to produce the large number of muons for the measurement, corresponding to about $\SI{2e11}{}$ detected decay positrons above some energy threshold. In Run 1 of the experiment, E989 gathered about $\SI{1e10}{}$ decay positrons for a precision comparable to the BNL result.


% Because the muon is much more massive than the electron, \amu is more sensitive to high energy scales as compared to the anomalous magnetic moment of the electron, which has already been measured to high precision. 


The experiment principle is summarized as follows: Polarized muons are injected into a highly uniform magnetic storage ring. The injected muons will orbit around the ring at the cyclotron frequency, and their spins will turn at the Larmor precession frequency. The difference between these two frequencies, \wa, is directly proportional to \amu and the magnetic field of the ring. \amu is then determined by measurements of the magnetic field and \wa. The magnetic field is measured by many nuclear magnetic resonance (NMR) probes located around the ring which monitor the field at all times. Separately a set of 17 NMR probes periodically measure the magnetic field where the muons live. \wa is measured by taking advantage of the parity violating nature of the weak decay of the muon, which produces a correlation between the muon spin at the time of the decay and the decay positron direction and momentum. The positrons, which spiral inward, are measured by calorimeters placed on the inside of the storage ring. The number of positrons above some energy threshold is modulated by \wa, which can be extracted.


In order to collect the large number of decay positrons necessary for the statistical goal, there are elements of the experiment designed to store a large number of muons in the ring at a time for a long period of data taking. For Run 1 this was about 10,000 injected muons at a time, which decay with a lifetime of $\SI{64.4}{\micro s}$, where data was gathered for approximately 10 muon lifetimes. Muons are produced by the accelerator complex and injected into the \gmtwo storage ring through a very small aperture. The injected muons are kicked onto the right orbit within the ring using a magnetic kicker. Muons are focused vertically within the ring using electrostatic quadrupoles around the ring. Because of all these elements, and the spread in phase space of the injected muons, the muon beam will move dynamically within the storage volume. This turns out to be important when extracting \wa or measuring the magnetic field, since it's technically the field that the muons see that we care about. There is a detector system involving straw trackers which facilitate the measurement of the muon beam dynamics.


The main part of my thesis involves the fitting of the positron data and the extraction of the \wa frequency. I use a specific analysis technique called the Ratio Method which has the advantage of reducing slowly time varying effects in the data to provide a more robust estimate of \wa. It involves splitting and time shifting the data in such a way that by dividing a portion of the data over the rest, the exponential nature of the number of counts vs time is eliminated and slow effects are reduced. Many checks are done on the fitting in order to make sure the fit behaves properly, time varying effects have been sufficiently accounted for, subsets of the data corresponding to individual calorimeters or run conditions are understood, etc. Many systematic effects which would pull the value of \wa are studied, such as pileup in the detectors, energy response of the detectors, changing beam dynamics, etc. Only after everything has been checked extensively can the final value of \wa be believed. 


The second part of my thesis focuses on the track fitting, in order to determine the muon beam distribution and dynamics. The track reconstruction as a whole involves separate stages to take in the hits seen in the trackers, form the hits into individual tracks, and then extrapolate them back into the storage ring. My relevant work involves the fitting stage, where I take a single set of hits that have been designated as belonging to a single track, and I fit the best track to those hits. This process involves propagation code contained within a Geant4 simulation, and a \chisq minimization routine. The propagation code involves propagating particles in the simulation forwards or backwards along their average trajectories, and calculating transport and error matrices for the track which describe the changes and spread in track parameters respectively. These matrix objects are then plugged into a relatively simple \chisq minimization algorithm which incorporates material correlations between measurement planes, calculates a covariance matrix for the track, and finally produces a best fit track which typically converges after three iterations of the fitting procedure.


As an experiment, we expect to publish the results from the Run 1 analysis sometime in late 2019. At this point the \wa extraction should be complete and compared between the different analyzers, the field analysis should be complete, and the combination between the two finished to provide the final value of \amu. The result should be comparable to the BNL level result, which should give confidence in the previously measured discrepancy or resolve it. By 2021 the necessary statistics should be captured for a four times more precise measurement.



\section*{Outline}

\section{Introduction}

\begin{itemize}
	\item{Magnetic moments of particles}
	\item{Standard Model contributions to $a_{\mu}$}
	\item{QED}
	\item{Electroweak}
	\item{Hadronic}
	\item{Experimental value and discrepancy with theory}
	\item{Beyond the Standard Model}
\end{itemize}

\section{Principle Techniques of E989}

\begin{itemize}
	\item{Measuring the precession frequency}
	\item{Measuring the magnetic field}
	\item{Production and injection of polarized muons}
	\item{Storage of muons}
	\item{Muon beam dynamics}
	\item{Corrections to the precession frequency}
\end{itemize}

\section{Detector Systems}

\begin{itemize}
	\item{Auxiliary Detectors}
	\item{Calorimeters}
	\item{Laser calibration system}
	\item{Straw trackers}
\end{itemize}

\section{Track Reconstruction and Analysis}

\begin{itemize}
	\item{Track finding}
	\item{Track fitting}
	\item{Track extrapolation}
	\item{Muon beam measurements}
\end{itemize}

\section{Precession Frequency Analysis}

\begin{itemize}
	\item{Hit reconstruction}
	\item{Fitting with the Ratio Method}
	\item{Stability vs fit start time}
	\item{Individual calorimeter fits}
	\item{Systematic studies vs pileup} 
	\item{Systematic studies vs gain}
	\item{Systematic studies vs beam dynamics} 
 	\item{Systematic studies vs muon losses} 
	\item{Systematic studies vs other} 
\end{itemize}

\section{Conclusion}

\begin{itemize}
	\item{Summary of systematic errors}
	\item{Final value of $a_{\mu}$}
	\item{Looking forward to the next runs}
\end{itemize}


\end{document}
