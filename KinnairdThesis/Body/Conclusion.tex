%!TEX root = ../thesis.tex

\thispagestyle{myheadings} % should I be including this at the top of every section page??

\chapter{Conclusion}
\label{chapter:Conclusion}


Experiment E821 measured the anomalous magnetic moment of the muon, \amu, to a relative uncertainty of 540 parts per billion. The measured value corresponds to a discrepancy of three to four standard deviations from the Standard Model theoretical value, depending on the chosen theory. In order to verify this discrepancy, a new experiment based upon the same principles has been undertaken to measure the same quantity with greater precision. Fermilab Muon \gmtwo, E989, measures \amu by measuring the spin precession frequency of muons within a magnetic storage ring, and the strength of the magnetic field that the muons experience. The former is done by counting the number of decay positrons observed in electromagnetic calorimeters above an energy threshold, while the latter is done using NMR probes in and around the muon storage region. 


Straw trackers assist both measurements by measuring muon beam dynamics which directly impact both the precession frequency measurement and the distribution of muons within the measured magnetic field. In \chapref{chapter:TrackReconstruction} this dissertation presented the track fitting algorithm used in the reconstruction of decay positron tracks, necessary for reconstructing muon beam decay vertices. The track fitting method, Geane, propagates positrons in the full E989 Geant4 simulation, with the advantage of direct access to the geometry, material, and non-uniform magnetic field present within the tracker region. Transport matrices, error matrices, and predicted parameter vectors are generated which are used in a global \chisq minimization algorithm, to produce optimal state vectors at the entrance to the tracker. These state vectors are extrapolated back into the storage region to approximate muon decay vertices, from which the muon distribution and beam motion can be derived. The beam motion was observed to be coherent, such that the muons as a whole oscillate within the storage region. This coherent motion has been well characterized both radially and vertically. These radial and vertical oscillations introduce modulations on top of the \wa signal seen in the decay positron time spectra. Due to damaged quadrupole resistors in Run~1, the frequency of these oscillations was found to be changing over time. Both the modulations and the changing frequency effect were included in the precession frequency analysis. Lastly, the equilibrium vertical distribution of the muons ties directly into the pitch correction which shifts the measured \wa frequency by $\mathcal{O}(\SI{100}{ppb})$. Preliminary analysis of the 60h dataset has determined a value of $-160 \pm 15$\xspace ppb. Numerous plots regarding the beam distribution and dynamics can be found in \secref{sec:MuonBeamMeasurements}. 


Muon decay is a self-analyzing process, meaning that decay positrons retain muon spin information. The correlation between the emission direction of high energy decay positrons and the muon spin at the time of the decay provides the signal with which to measure the muon spin precession frequency. Decay positrons above an energy threshold are counted and put into a time histogram, from which the \wa oscillation can be extracted. The time spectrum is corrected for gain variations and the pileup background which distort the \gmtwo signal. The precession frequency in this analysis is extracted using a technique called the Ratio Method, detailed in \chapref{chapter:wa}. The Ratio Method works by splitting the positron decay time spectrum into four subsets, time-shifting two of them, and taking the ratio of the difference and sum of the shifted and un-shifted datasets respectively. This procedure removes the muon decay exponential from the data while also reducing any slowly and smoothly varying effects present in the data. Precession frequency extraction analyses for four near-final Run~1 datasets were presented with full systematic error evaluations. Fit functions including terms which account for various beam dynamics effects have been applied to the data with success. The integrity of the fits have been checked by splitting up the data in numerous ways, and verifying that fit results remain consistent regardless of detector number, fit start time, energy threshold, bunch number, etc. Systematic errors were evaluated with respect to the assumptions made regarding beam dynamics terms, the subtraction of the pileup background, treatment of the gain variations, and more. The systematic errors are well understood, with independent groups currently working on improving the largest outstanding systematic errors. Preliminary estimates of the E-field corrections are $-519 \pm 27$, $-463 \pm 36$, and $-467 \pm 20$\xspace ppb, for the 60h, 9d, and Endgame datasets respectively. Summary tables for the systematic errors and final extracted precession frequency values can be found in Tables~\ref{tab:FinalSystematicErrors} and \ref{tab:FinalResults}. The total Run~1 precession frequency error determined in this analysis is \SI{\TotalCorrErr}{ppb}, where the error is statistics dominated.


The expected error in the magnetic field measurement for Run~1 is $\mathcal{O}(\SI{140}{ppb})$. Combined with the total precession frequency error determined in this analysis, the error on \amu is expected to be $\mathcal{O}(\SI{500}{ppb})$, comparable to the uncertainty in the E821 measurement. For the final Run~1 production datasets with the last data quality cuts and gain improvements, the final errors are expected to improve only slightly. An independent measurement of \amu that is statistically consistent with the E821 result would go a long way towards increasing the confidence in the discrepancy between theory and experiment. The Run~1 publication is expected to be complete sometime in 2020. Data has already been gathered for Run~2, Run~3 has just begun in the late fall of 2019, and Run~4 is planned to run from late 2020 to the middle of 2021. With the rate improvements seen in Run~2 and the expected increases for Run~3 and Run~4, the target uncertainty goal of \SI{140}{ppb} is a likely reality. Assuming the same central value for \amu is measured as was done in the previous experiment, the statistical significance of the discrepancy between the theory and experiment would be pushed over five standard deviations, providing strong evidence for the existence of new physics, and constraining any explanatory models.

% Read David Flay's talk from collaboration meeting for statistics numbers for Run 2 and 3 if I want to add those in here
% 2.2x BNL for Run 2, but DQC expected to be a lot better and more stable quad/kicker conditions
% as a reminder Run 1 was ~ 2x BNL but bad conditions reduced that to about ~1x
% Run 3 goal is to double Run 2 stats - 4x BNL






