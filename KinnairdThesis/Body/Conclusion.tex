%!TEX root = ../thesis.tex

\thispagestyle{myheadings} % should I be including this at the top of every section page??

\chapter{Conclusion}
\label{chapter:Conclusion}


Experiment E821 measured the anomalous magnetic moment of the muon to a relative uncertainty of 540 parts per billion, corresponding to a discrepancy of three to four standard deviations from the Standard Model theoretical value. In order to confirm or deny this discrepancy, a new experiment has been undertaken to measure the same quantity. E989 measures \amu by measuring the spin precession frequency of muons within a magnetic storage ring and the strength of the magnetic field within the storage region. The former is done by counting the number of decay positrons observed in electromagnetic calorimeters above an energy threshold, while the latter is done using NMR probes in and around the muon storage region. 



Straw trackers assist both measurements by measuring muon beam dynamics which directly impact both the precession frequency measurement and the distribution of muons within the measured magnetic field. In \chapref{chapter:TrackReconstruction} this dissertation presented the track fitting algorithm used in the reconstruction of decay positron tracks necessary for reconstructing muon beam decay vertices. The track fitting method, Geane, propagates positrons in the full E989 Geant4 simulation with the advantage of direct access to the geometry, material, and non-uniform magnetic field present within the tracker region. Transport matrices, error matrices, and predicted parameter vectors are generated which are then used in a global \chisq minimization algorithm which produces an optimal state vector at the entrance to the tracker. This state vector is then extrapolated back into the storage region to approximate muon decay points, from which the muon distribution and beam motion can be derived. The coherent beam motion has been well characterized both radially and vertically. Radially, the beam is seen to oscillate inwards and outwards which introduces a modulation on top of the \wa signal seen in the decay positron spectra, while the oscillation of the vertical width does the same. The characterization of the changing frequency of this coherent beam motion due to damaged quadrupole resistors in Run~1 was shown, and similarly included in the precession frequency analysis. Numerous plots regarding the beam dynamics can be found in \secref{sec:MuonBeamMeasurements}. The equilibrium vertical distribution of the muons as evaluated by the tracker ties directly into the pitch correction which shifts the measured \wa frequency on the order of over \SI{100}{ppb}. Preliminary analysis of the 60h dataset has given a value of $-160 \pm 15$\xspace ppb.



Muon decay is a self-analyzing process, meaning that decay positrons retain muon spin information. The correlation between the emission direction of high energy decay positrons and the muon spin at the time of the decay provides the signal with which to measure the muon spin precession frequency. Decay positrons above an energy threshold are counted and put into a time histogram, on which the \wa oscillation can be observed. The time spectra is corrected for gain variations and the pileup background which distort the \gmtwo signal. The precession frequency is extracted using an analysis technique called the Ratio Method, detailed in \chapref{chapter:wa}. The Ratio Method works by splitting the positron decay time spectra into four subsets, time-shifting two of them, and taking the ratio of the difference and sum of the shifted and un-shifted datasets respectively. This procedure removes the muon decay exponential from the data while also reducing any slowly varying effects present in the data. Fit functions including terms which account for various beam dynamics effects have been applied to the data with success. The integrity of the fits have been checked by splitting up the data in numerous ways, and veryifing that fit results remain consistent regardless of detector number, fit start time, energy threshold, bunch number, etc. Precession frequency extraction analyses for four near-final Run~1 datasets were presented with full systematic error evaluations. Systematic errors were evaluated with respect to the assumptions made regarding beam dynamics terms, the subtraction of the pileup background, treatment of the gain variations, and more. The systematic errors are well understood, with independent working groups currently working on improving the largest systematic errors. Summary tables can be found in Tables~\ref{tab:FinalSystematicErrors} and \ref{tab:FinalResults}. The total Run~1 precession frequency error determined in this analysis was \SI{\TotalCorrErr}{ppb}, where the error is statistics dominated.


The expected error in the magnetic field measurement for Run~1 is $\mathcal{O}(\SI{140}{ppb})$. That combined with the total precession frequency error determined in this analysis would result in an error on \amu of $\mathcal{O}(\SI{500}{ppb})$, comparable to the uncertainty in the E821 measurement. For the final Run~1 production datasets with the last data quality cuts and gain improvements, the final errors are expected to improve slightly. An independent measurement of \amu that is statistically consistent with the E821 result would go a long way towards increasing the confidence in the discrepancy between theory and experiment. The Run~1 publication is expected to be complete sometime in 2020. Data has already been gathered for Run~2 of E989 and Run~3 has just begun in the late fall of 2019, with Run~4 planned for 2021. With the rate improvements seen in Run~2 and the expected increases for Run~3 and Run~4, the target uncertainty goal of \SI{140}{ppb} is a likely reality. Assuming the same central value for \amu is measured as was done in the previous E821 experiment, then the statistical significance of the discrepancy between the theory and experiment would be pushed over five standard deviations, providing very strong evidence for the existence of new physics and constraining any explanatory models.

% Read David Flay's talk from collaboration meeting for statistics numbers for Run 2 and 3 if I want to add those in here
% 2.2x BNL for Run 2, but DQC expected to be a lot better and more stable quad/kicker conditions
% as a reminder Run 1 was ~ 2x BNL but bad conditions reduced that to about ~1x
% Run 3 goal is to double Run 2 stats - 4x BNL






