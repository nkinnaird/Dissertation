%!TEX root = ../thesis.tex

\thispagestyle{myheadings} % should I be including this at the top of every section page??

\chapter{Conclusion}
\label{chapter:Conclusion}


Experiment E821 measured the anomalous magnetic moment of the muon to a relative uncertainty of 540 parts per billion, corresponding to a three to four standard deviation from the Standard Model theoretical value. In order to confirm or deny this discrepancy, a new experiment has been undertaken to measure the same quantity.


E989 measures \amu by measuring the spin precession frequency of muons within a magnetic storage ring and the strength of the magnetic field within the storage region. The former is done by counting the number of decay positrons observed in electromagnetic calorimeters above an energy threshold, while the latter is done using NMR probes in and around the muon storage region. Straw trackers assist both measurements by measuring muon beam dynamics which directly impact both the precession frequency measurement and the distribution of muons within the measured magnetic field. This dissertation has presented the track fitting algorithm used in the reconstruction of decay positron tracks necessary for reconstructing the muon beam decay vertices. The track fitting method, GEANE, propagates positrons in the full E989 Geant4 simulation with the advantage of direct access to the geometry, material, and non-uniform magnetic field present within the tracker region. Transport matrices, error matrices, and predicted parameter vectors are generated which are then used in a global \chisq minimization algorithm which produces an optimal state vector near the start of the tracker. This procedure was used successfully as described in \chapref{chapter:TrackReconstruction}, which revealed a wealth of beam dynamics information vital to the experiment.


The precession frequency was extracted using an analysis technique called the Ratio Method. The Ratio Method takes the ratio of time-shifted decay positron spectra in order to remove the decay exponential along with slowly varying effects in the data. Precession frequency extraction analyses for four near-final Run~1 datasets were presented with full systematic error evaluations in \chapref{chapter:wa}. The systematic errors are well understood, with independent working groups currently working on improving the largest systematic errors. The total Run~1 precession frequency error determined in this analysis was $\SI{491.2}{ppb}$, where the error is statistics dominated. 


The expected error in the magnetic field measurement for Run~1 is $\mathcal{O}(\SI{140}{ppb})$. That combined with the total precession frequency error determined in this analysis would result in an error on \amu of $\mathcal{O}(\SI{510}{ppb})$, comparable to the uncertainty in the E821 measurement of \SI{540}{ppb}. For the final Run~1 production datasets with the last data quality cuts and gain improvements, the final errors are expected to improve slightly. An independent measurement of \amu that is statistically consistent with the E821 result would go a long way towards increasing the confidence in the discrepancy between theory and experiment of nearly four standard deviations. The Run~1 publication is expected to be complete in the spring of 2020. Data has already been gathered for Run~2 of E989 and Run~3 will begin in the late fall of 2019, with Run~4 planned for 2021. With the rate improvements seen in Run~2 and the expected increases for Run~3 and Run~4, the target uncertainty goal of \SI{140}{ppb} is a likely reality. Assuming the same central value for \amu is measured as was done in the previous E821 experiment, then the statistical significance of the discrepancy between the theory and experiment would be pushed over five standard deviations, confirming the existence of new physics.






