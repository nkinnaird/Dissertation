%!TEX root = ../thesis.tex

\thispagestyle{myheadings}

\begin{center}
  \textbf{{\LARGE Nicholas Kinnaird}} \\
  \small 512-922-0534 $\vert$ nickkinn@bu.edu $\vert$ 1101 Iroquois Ave Apt 1401, Naperville, IL 60563
% \end{center}

\vspace{0.2in}
\begin{tabular*}{\linewidth}{@{\extracolsep{\fill}}lr}
  \multicolumn{2}{l}{\textbf{{\Large Education}}} \\
% \hrule{\linewidth}{3pt} \\
\\
\hline
\\

  {\bf Ph.D. in Physics} 		& Expected 2020 \\
  Boston University      		& Boston, MA \\
  						 		& \\ 
	 \multicolumn{2}{R{\linewidth}}{Dissertation: Muon Spin Precession Frequency Extraction and Decay Positron Track Fitting in Run 1 of the Fermilab Muon \gmtwo Experiment} \\
  						 		& \\ 
  {\bf M.A. in Physics}  		& 2016 \\
  Boston University      		& Boston, MA \\
  						 		& \\ 
  {\bf B.S. in Physics}  		& 2013 \\
  University of Texas at Austin & Austin, TX \\
  						 		& \\ 
  {\bf B.S. in Mathematics}     & 2013 \\
  University of Texas at Austin & Austin, TX \\
  						 		& \\ 
\end{tabular*}

\vspace{0.2in}
\begin{tabular*}{\linewidth}{@{\extracolsep{\fill}}l}
  \multicolumn{1}{l}{\textbf{{\Large Projects}}} \\
\\
\hline
\end{tabular*}

\begin{itemize}
	\item{{\bf Track Fitting:} Developed code to fit decay positron tracks in the E989 experiment. The algorithm was implemented into a modular C++ framework to interface with upstream and downstream modules of the track reconstruction code. The algorithm fit tracks by propagating particles in a Geant4 simulation and doing a \chisq minimization with track representation matrix objects.}
	
	\item{{\bf Frequency Extraction:} Performed frequency analysis of Run~1 data of the E989 experiment. The project involved writing code to extract a frequency from histograms to high precision by fitting functions. The code was generalized to allow fitting different subsets of the data, or with different fit parameters, in order to systematically evaluate the data and verify the integrity of the results.}
	
	\item{{\bf Magnetic Field Analysis:} Measured and analyzed external magnetic fields in the E989 experimental hall before construction using a fluxgate magnetometer. Simulated and analyzed the E989 magnetic ring storage and fringe field in Opera 2D and Opera 3D.}
\end{itemize}


\vspace{0.2in}
\begin{tabular*}{\linewidth}{@{\extracolsep{\fill}}ll}
  \multicolumn{2}{l}{\textbf{{\Large Technical Skills}}} \\
% \hrule{\linewidth}{3pt} \\
\\
\hline
\\

  {\bf Programming Languages} & C++, C, Python, LaTex, Bash \\
  							  & \\
  {\bf Software} 		      & Mathematica, Opera 2D, Opera 3D, Geant4,  \\
  							  & ROOT, git, Paraview, Linux, OSX, Windows \\
  							  & \\
  {\bf General} 		      & Collaboration, Presentation, Technical Writing, \\
  							  & Data Analysis, Hardware Testing, Tutoring \\ 
  							  & \\ 					
\end{tabular*}


% TA for undergrads, undergraduate lab stuff
% APS 2018 conference


\end{center}
