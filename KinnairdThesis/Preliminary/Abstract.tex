%!TEX root = ../thesis.tex

% ABSTRACT


One of the few indications for new physics is the discrepancy between the theoretical and experimental values for the anomalous magnetic moment of the muon. There is a discrepancy of 3 to 4 standard deviations between theory and the last experimental measurement made at Brookhaven National Laboratory in 2001, which measured the muon magnetic anomaly \amu to 540 parts per billion (ppb). This discrepancy has been consistent for many years with ever improving theoretical calculations. In order to resolve or confirm this discrepancy experiment E989, Muon \gmtwo, is underway to measure \amu to 4 times higher precision at \SI{140}{ppb}. In Run~1 E989 gathered its first production data, consisting of approximately \SI{8e9}{} decay positrons above an energy threshold of \SI{1.7}{\GeV}.

This dissertation describes the experimental measurement, the detectors, the precession frequency extraction, and the track fitting of decay positrons in Run~1. The track fitting is done using a \chisq minimization algorithm to fit tracks propagated within a Geant4 reconstruction simulation including error propagation. The precession frequency is extracted using an analysis technique called the Ratio Method. The Ratio Method takes the ratio of time-shifted decay positron spectra in order to remove the decay exponential along with slowly varying effects in the data. Precession frequency extraction analyses for four near-final Run~1 datasets are presented with full systematic error evaluations. The total Run~1 precession frequency error determined in this analysis is \SI{\TotalCorrErr}{ppb}, where the error is statistics dominated. Combined with the expected error in the magnetic field measurement of \SI{140}{ppb}, the expected final error on \amu for Run~1 of E989 is $\mathcal{O}(\SI{500}{ppb})$, comparable to the previous measurement.






